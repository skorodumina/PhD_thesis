%%%%%%%%%%%%%%%%%%%%%%%%%%%%%%%%%%%%%%%%%%%%%%%%%%%%%%%%%%%%%%%%%%
%%%%%%%%%%%%%%%%%%%%%%%%%%%%%%%%%%%%%%%%%%%%%%%%%%%%%%%%%%%%%%%%%%
%                       ABSTRACT PAGE
%%%%%%%%%%%%%%%%%%%%%%%%%%%%%%%%%%%%%%%%%%%%%%%%%%%%%%%%%%%%%%%%%%
%%%%%%%%%%%%%%%%%%%%%%%%%%%%%%%%%%%%%%%%%%%%%%%%%%%%%%%%%%%%%%%%%%

\patchcmd{\abstract}{\null\vfil}{}{}{}
\renewcommand{\abstractname}{\LARGE\vspace{\baselineskip ~~~\\~~~\\~~~}{\bf\LARGE  Abstract}}

\cleardoublepage
\phantomsection
\addcontentsline{toc}{chapter}{{Abstract}}{}{}

\begin{abstract}




\thispagestyle{plain}

\doublespacing
\noindent
In this study, the process of $\pi^{+}\pi^{-}$ electroproduction off protons bound in deuterium nuclei is explored. The exploration is performed through the analysis of experimental data on electron scattering off the deuteron target, collected in Hall B at Jefferson Lab with the CLAS detector. As a main result, the set of integrated and single-differential cross sections of the reaction $\gamma_{v}p(n) \rightarrow p' (n')\pi^{+}\pi^{-}$ is obtained. The cross sections are extracted in the quasi-free regime, which implies that only events not affected by final state interactions are subject to selection. The measurements are performed in the kinematic region of the invariant mass $W$ from 1.3~GeV to 1.825~GeV and photon virtuality $Q^{2}$ from 0.4~GeV$^2$ to 1~GeV$^2$. Sufficient experimental statistics allows narrow binning, i.e. 25 MeV in $W$ and 0.05 GeV$^{2}$ in $Q^{2}$, while maintaining an adequate statistical uncertainty. The extraction of quasi-free cross sections is accompanied by the kinematic probing of FSI between the reaction final hadrons and the spectator neutron in the aforementioned exclusive channel. In this probing the distributions of missing quantities are used in order to investigate the relative spread of events with FSI along the reaction phase space, trace the difference of FSI manifestations in different reaction topologies, reveal details on alterations of the hadron momentum in FSI, and isolate FSI contributions of various final hadrons. The performed examination is also capable of retrieving information on some underlying FSI mechanisms, among which the process of resonance formation in the intermediate state of pion-neutron interactions is particularly remarkable.


\setcounter{page}{4}

\end{abstract}

