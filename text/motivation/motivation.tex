\chapter{Introduction}
\pagenumbering{arabic} 
\setcounter{page}{1}
\label{Sect:motiv}
%\refstepcounter{chapter}

\everypar{\looseness=-1}
Exclusive reactions of meson photo- and electroproduction off protons are intensively studied in laboratories all over the world as a very powerful tool for the investigation of nucleon structure and the principles of strong interaction. These studies include the extraction of various observables through the analysis of experimental data as well as subsequent theoretical and phenomenological interpretations of the extracted observables~\cite{Krusche:2003ik,Aznauryan:2011qj,Skorodumina:2016pnb}.


By now exclusive reactions off the free proton have been studied in considerable detail, and a lot of information on differential cross sections and different single and double-polarization asymmetries with almost complete coverage of the reaction phase-space is available. A large part of this information came from the analysis of data collected in Hall B at Jefferson Lab with the CLAS detector~\cite{Mecking:2003zu,CLAS_DB}.% and stored in the CLAS physics database~\cite{CLAS_DB}.


Meanwhile, reactions occurring in photon and electron scattering off nuclei are less extensively investigated, i.e. experimental information on these processes is scarce and mostly limited to inclusive measurements of total nuclear photoproduction cross sections~\cite{Mokeev:1995fy,Bianchi:1994ax,Ahrens:1986hn} and nucleon structure function $F_{2}$~\cite{Osipenko_2005_note,Osipenko:2005gt,Osipenko:2010sb}. 

The available inclusive data, however, exhibit some surprising peculiar features not fully elucidated over the years, which are now attracting significant scientific attention. Specifically, the nuclear photoproduction cross section (per nucleon) turns out to be less pronounced and damped in strength compared with the cross section off the free proton. This effect manifests itself differently depending on the invariant mass ($W$) range, i.e. the $\Delta(1232)$-resonance peak is damped, but still evident for all nuclei, however, the second resonance region becomes somewhat less pronounced and damped for the deuteron and strongly suppressed and structureless for all heavier nuclei. 

A similar effect is observed in the behavior of the nucleon structure function $F_{2}$, which in the case of the deuteron shows moderate damping and flattening~\cite{Osipenko:2005gt} and completely loses its structure, when measured off carbon~\cite{Osipenko:2010sb} (compared with the free proton structure function~\cite{Osipenko:2003bu}). A fact of particular interest is that the intensity of this effect increases as $Q^{2}$ grows, i.e. as $Q^{2}$ = 3~GeV$^{2}$ is reached, the structure function $F_{2}$ becomes almost flat even for the deuteron~\cite{Osipenko:2010sb}. These peculiar features cannot just be explained by the Fermi motion of nucleons in the nucleus and are thought to be an indication that nucleons and their exited states, bound inside the nuclear medium, may be subject to some modifications of their properties~\cite{Mokeev:1995fy,Bianchi:1994ax,Ahrens:1986hn,Krusche:2004xz,Noble:1980my}. 



This phenomenon, which is still not fully understood, generates lots of debates among scientists, triggering efforts to describe the processes that happen in reactions off bound nucleons. These studies rely heavily on experimental data, which at the moment are mostly limited to inclusive measurements~\cite{Mokeev:1995fy,Bianchi:1994ax,Ahrens:1986hn,Osipenko_2005_note,Osipenko:2005gt,Osipenko:2010sb} and lack information on exclusive reactions. This information, however, is crucial, since various exclusive channels have different energy dependencies and different sensitivity to reaction mechanisms. This situation causes a strong demand for exclusive measurements off bound nucleons, and the deuteron, being the lightest and weakly-bound nucleus, is the best target for initiating these efforts.


This study represents a thorough exploration of the process of charged double-pion electroproduction off protons bound in deuterium nuclei. The exploration has been performed through the analysis of experimental data on electron scattering off the deuteron target, collected with the CLAS detector~\cite{Mecking:2003zu}. The description of the detector and target setup is given in Chapter~\ref{Chapt:experiment} together with information on the data format and the overall analysis structure. 

The experimental measurements have been performed in the second resonance region, where the double-pion production plays an important role, i.e. the channel opens at the double-pion production threshold at $W \approx 1.22$~GeV, contributes significantly to the total inclusive cross section for $W \lesssim 1.6$~GeV, and starts to dominate all other exclusive channels for $W \gtrsim 1.6$~GeV.


In general, experimental identification of exclusive multi-particle final states is a rather sophisticated task, which requires certain analysis techniques to be elaborated and established. This was carried out over the last twenty years as different studies of double-pion production off the free proton were being performed~\cite{Rip_an_note:2002,Ripani:2002ss,Fed_an_note:2007,Fedotov:2008aa,Isupov:2017lnd,Golovach,Arjun,Fed_an_note:2017,Fed_paper_2018}, and currently a solid framework for such studies is in place. For this particular study, focused on the $N\pi\pi$ final state, this framework laid the foundation. However, the deuteron as a target introduces some specific issues, which are external to the free proton data analysis and originate from (a) Fermi motion of the initial proton and (b) complex effects of the final state interactions due to the presence of a spectator nucleon. This caused some difficulties that were encountered and needed to be overcome during the analysis and, therefore, in this study special attention is paid to detailed description of these issues.

As this study represents the first attempt of detailed investigation of the double-pion exclusive reaction occurring off nucleons in nuclei, its objectives are multifaceted. 



The main goal of this study is to obtain the set of integrated and single-differential cross sections of the reaction $\gamma_{v}p(n) \rightarrow p' (n')\pi^{+}\pi^{-}$. The cross section measurements are performed in the kinematic region of the invariant mass $W$ from 1.3~GeV to 1.825~GeV and photon virtuality $Q^{2}$ from 0.4~GeV$^2$ to 1~GeV$^2$. Sufficient experimental statistics allows narrow binning, i.e. 25~MeV in $W$ and 0.05~GeV$^2$ in $Q^2$, while maintaining an adequate statistical uncertainty. The cross sections are extracted in the quasi-free regime, which implies that only events not affected by final state interactions were selected.


The whole enterprise of the cross section extraction is presented in Chapters~\ref{Sect:select} through~\ref{Sect:uncert}, which encompass the selection of quasi-free events, the cross section calculation framework, the description of the corrections applied to the cross sections, as well as the procedures of normalization verification and uncertainty estimation. 

To exploit opportunities offered by this experiment in their full capacity, the main analysis is accompanied by the complementary examination of FSI effects in the investigated exclusive channel, which is presented in Chapter~\ref{Sect:fsi_discuss}. The main focus of this examination is set on events affected by FSI, which were lacking attention throughout the main analysis being attributed to the background. These events, meanwhile represent a fruitful ground for probing FSI and revealing their features and manifestations.

Another objective of this study stems from the Fermi motion that initial protons undergo in deuterium nuclei. The fact that the initial proton is not at rest introduces several unaccustomed peculiarities into the analysis, which were not relevant for free proton studies, such as Fermi smearing of some kinematic quantities (e.g. $W$ and missing masses), blurring of the boundaries of the $W$ versus $Q^{2}$ distribution, alterations in the common procedure of the Lab to CMS transformation, etc. To deal with these issues, special methods and techniques have been developed during this study, which enrich the conventional analysis framework elaborated in numerous free proton studies~\cite{Rip_an_note:2002,Ripani:2002ss,Fed_an_note:2007,Fedotov:2008aa,Isupov:2017lnd,Golovach,Arjun,Fed_an_note:2017,Fed_paper_2018}.


As a matter of fact, effects of the initial proton motion turned out to be tightly interwoven with many analysis aspects. For this reason, it was found hard to address them in an isolated Chapter without introducing much repetition. Therefore, the description of the analysis peculiarities that originate from the Fermi motion is scattered throughout the thesis. Some of them are also addressed in a separate study~\cite{twopeg-d}, which accompanies this analysis.  



It is also worth mentioning that this study benefits from a very fortunate circumstance; namely from the fact that the corresponding cross sections of the same exclusive reaction off the free proton have been recently extracted from CLAS data~\cite{Fed_an_note:2017,Fed_paper_2018}. These free proton measurements were performed under the same experimental conditions as in this study, including the beam energy value and the target setup. For the majority of $(W,~Q^{2})$ points, the statistical uncertainty combined with the model dependent uncertainty ($\delta_{\text{stat,~mod}}^{\text{tot}}$) is on a level of $\sim$1\%-3\% for the free proton integral cross sections and on a level of $\sim$4\%-6\% for the quasi-free integral cross sections obtained in this study. Being performed in the same experimental configuration, both measurements have identical binning in all kinematic variables and similar inherent systematic inaccuracies. 

The free proton study~\cite{Fed_an_note:2017,Fed_paper_2018}, therefore, naturally sets the standard for this particular study, being used as a reference point for many analysis components. This unique advantage allows not only the reliability verification of those analysis aspects that are similar for experiments of free and bound protons, but also for a deeper understanding of those that differ. The latter include the effects of initial proton motion and final state interactions.


Additionally, a direct comparison of the cross sections extracted in this study with their free proton analogues from Ref.~\cite{Fed_an_note:2017,Fed_paper_2018} may represent a promising further step in the investigation of the double-pion exclusive channel. Such comparison can provide experimentally the best possible opportunity to explore distinctions between the $\pi^{+}\pi^{-}$ electroproduction off protons in deuterium and the corresponding reaction off free protons. In this way, a condition-independent estimation of contribution from events with FSI to the total number of reaction events can be accomplished along the entire reaction phase-space. Based on this, other potential reasons that may contribute to the difference between the cross section sets can be explored, which includes possible in-medium modifications.




