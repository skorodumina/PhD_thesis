\chapter{Some other issues}
\label{Sect:issues}


\section{The cross section beam energy dependence}

The $\varphi$-integrated virtual photoproduction cross section $\sigma_{v}$ can be decomposed into the combination of the structure functions~\cite{twopeg,Skorodumina:2016pnb},\vspace{-1.25em}
\begin{equation}
\sigma_{v} = \sigma_{T} + \varepsilon_{L}\sigma_{L}\text{~~~with~~~}\varepsilon_{L}=\frac{Q^{2}}{\nu^{2}}\varepsilon_{T}\\[-10pt],\label{eq:beam_en_dep}
\end{equation}
where $\sigma_{T}$ and $\sigma_{L}$ are the transverse and longitudinal structure functions, respectively, and $\varepsilon_{L}$ the longitudinal polarization of the virtual photon with $\varepsilon_{T}$ given by~Eq.~\eqref{polarization}.

Being decomposed in this way, the cross section $\sigma_{v}$ has a specific beam energy dependence, which is incorporated into the coefficient $\varepsilon_{L}$. The structure functions themselves, meanwhile, do not depend on the beam energy. A single experiment conducted with a certain beam energy allows for the extraction of $\sigma_{v}$ as a whole without accessing the separate structure functions. Thus, the beam energy dependence turns out to be implicitly incorporated into the extracted cross sections.

Although the experiment is conducted with a fixed value of the laboratory beam energy, the actual energy of the incoming electron involved in the reaction turns out to alter and differ from the fixed laboratory value due to  (i) the radiative effects that electrons undergo and (ii) the Fermi motion of the target proton. As a consequence, the extracted cross section cannot be associated with a distinct value of the electron beam energy, and this may complicate the interpretation of the results. Below these issues are addressed in more detail. 

\begin{itemize}
\item [(i)] The incoming and scattered electrons can emit photons thus reducing their energy. Due to the change of the incoming electron energy, the extracted cross sections correspond to the superposition of various beam energies. The correction due to this effect is included into the radiative corrections (see~Sect.~\ref{Sect:rad_corr}).
\item [(ii)] The experiment off the moving proton with fixed laboratory beam energy corresponds to that off the proton at rest performed with varying effective beam energies~\cite{twopeg-d}. As a result, the extracted cross sections off moving protons are convoluted with the dependence of the quantity  $\varepsilon_{L}$ on the beam energy (see Eq.~\eqref{eq:beam_en_dep}). A study in Ref.~\cite{twopeg-d}, however, proves that this effect has an insignificant influence on the cross section. The correction due to this effect (which is negligible anyway) is automatically included into the procedure of unfolding the effects of the target motion (see Sect.~\ref{Sect:fermi_corr})\footnote[1]{Note that the radiative effects decrease the beam energy, while the Fermi motion leads to a symmetrical spread of the effective beam energy around the laboratory value.}. 
\end{itemize}


Being corrected, the cross sections extracted in this analysis may be assigned to the distinct value of the laboratory beam energy of $E_{beam} = 2.039$~GeV.


\section{Off-shell effects}

\everypar{\looseness=-1}
The target proton is bound in the deuterium nucleus and thus undergoes nucleon-nucleon interactions. The nucleon mass, however, is thought to be an interaction-dependent quantity, i.e. the nucleon's physical mass in a nucleus is smaller than that of a free nucleon~\cite{Noble:1980my}. In other words, the target proton bound in the deuteron is off-shell, which means that its four-momentum squared is not equal to its mass squared.

In the study~\cite{Ye_Tian:2017}, which aimed at $\pi^{-}$ electroproduction off the neutron in deuterium, the impact of the off-shell effects on the measured cross sections was shown to be marginal. In this study the off-shell effects are ignored.





